\section{はじめに}

WebやIoT機器などから発信されるフロー情報が巷にはあふれている。
フロー情報とはニュースや天気予報、SNSなどリアルタイムに常に流れていく情報のことである。

フロー情報を視覚化する手法として\textgt{ダッシュボード}(図\ref{dashing})がある。
ダッシュボードとは単一の画面に情報を並べて表示するもので、センサの値や株価などのフロー情報をひと目で把握するのに非常に便利である。
ダッシュボードには様々な製品やサービスが存在し、多くの組織で利用されている。
フロー情報のひとつである「人間の感情や現在の状況」というものをアウトプットする場としてテキストチャットやSNSが多くの人に利用されているが、
時間順に投稿をリストするタイムライン表示は投稿したものが流れていってしまったり投稿数の多い人ばかりが目立ってしまったりするという問題がある。

一般的にダッシュボードに表示される情報に加えて、人や環境の状態も単一の画面に表示する視覚化システム「わかるらんど」を提案する。

\begin{figure}[h]
\centering
\includegraphics[width=7cm]{images/dashing.eps}
\caption{ダッシュボード}
\label{dashing}
\end{figure}