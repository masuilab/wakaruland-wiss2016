\section{議論}

\subsection{チャットシステムとしての利用}
講義やコンファレンスではチャットが利用されることが多いが、わかるらんどを利用することでチャットシステムが抱える問題を解決できる。
チャットシステムには3つの問題があると考える。

\begin{itemize}
\item 同時に多数が投稿するとすぐに流れていってしまう
\item 投稿数の多い人が目立ってしまう
\item 投稿しない人は全く投稿しない
\end{itemize}

わかるらんどは全員の最新の投稿を表示するので、

\begin{itemize}
\item 短時間に多くの人が投稿しても流れて見えなくなってしまうことがない
\item 投稿数が多いからといって目立つわけではない
\item 投稿のハードルが低い(「なるほど」とか「そうかも」みたいなクソリプでもいい)
\end{itemize}

わかるらんどのインタフェースは長い文章を投稿するのに適していないのでわかるらんど上で議論を行うことは難しい。
わかるらんどは参加者の感情をひと目で把握できるものであって、議論をするためのものではない。
発表の場合は最後に質問や議論の時間があるので議論はその時に行えばよい。
そもそも人が発表をしているときはチャットで議論なんてしてないで話を聞くべきである。

\subsection{実際の行動に基づく投稿}

別の作業を行っていてわかるらんどへの投稿をしたいときにブラウザを開いてスタンプを押さなければならないのが面倒である。
ボタンやテンキーなど専用の入力装置も作ることができるが、投稿できるものが限られている。
自分が心のなかで「なるほど」と思ったらわかるらんどに「なるほど」と投稿したい。
人間の行動に基づいて、膝を打ったら「なるほど」、首を捻ったら「わからん」などと投稿できたら嬉しい。

\subsection{表示する情報リストの作成}

\begin{itemize}
\item 100人参加のコンファレンスで全員のリストを作るのが大変
\item タプルスペースを部屋にすればできるが柔軟性に欠ける
\item いらない人を省きたい
\item 別のタプルスペースの情報を追加したい
\end{itemize}